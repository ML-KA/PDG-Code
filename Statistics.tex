% \begin{savequote}[75mm]
% Nulla facilisi. In vel sem. Morbi id urna in diam dignissim feugiat. Proin molestie tortor eu velit. Aliquam erat volutpat. Nullam ultrices, diam tempus vulputate egestas, eros pede varius leo.
% \qauthor{Quoteauthor Lastname}
% \end{savequote}
\chapter{Statistics}
Statistics is a branch of mathematics dealing with the collection, organization, analysis, interpretation and presentation of data.
\href{https://en.wikipedia.org/wiki/Statistics}{wikipedia}

\section{Probability}
TODO: Probability (general + simple), CDF, Variance, Fisher-Information etc.
\subsection{$L_p$-Space for Random-Variables}
The $L_p$-Norm for Random-Variables $X$, where $\mathbb{E}|X|^p < \infty$, is defined through:
\begin{align*}
	||X||_p:=(\mathbb{E}[|X|^p])^{\frac{1}{p}}
\end{align*}
\href{http://www2.stat.duke.edu/courses/Fall18/sta711/lec/wk-05.pdf}{lecture}

\subsection{Jensens-Inequality for Random Variables}
If $\phi$ is a konvex function and $X$ a Random-Variable, then
\begin{align*}
	\phi(\mathbb{E}X) \leq \mathbb{E}\phi(X)
\end{align*}
\href{https://en.wikipedia.org/wiki/Jensen%27s_inequality}{wikipedia}

\section{Distributions}
In this section, $X$ denotes a Random Vairable and $f$ the density-function.\\
TODO: More common distributions
\subsection{Normal Distribution}
If $X \sim {\mathcal {N}}(\mu ,\sigma ^{2})$ for ${\displaystyle \mu \in \mathbb {R}}$ and $\sigma ^{2} > 0 \in \mathbb {R}$, then:
\begin{align*}
	f(x) &= {\displaystyle {\frac {1}{\sqrt {2\pi \sigma ^{2}}}}e^{-{\frac {(x-\mu )^{2}}{2\sigma ^{2}}}}}\\
	\mathbb{E}X &= \mu \\
	Var[X] &= \sigma^2
\end{align*}
\href{https://en.wikipedia.org/wiki/Normal_distribution}{wikipedia}

\subsection{Normal Distribution (Multivariate)}
If $X \sim {\mathcal {N}}(\mu ,\Sigma)$ for $\mu \in \mathbb {R}^k$ and $\Sigma \in \mathbb {R}^{k \times k}$ with $\Sigma$ being positve semi-definite, then:
\begin{align*}
	f(x) &= \operatorname {det} (2\pi {\boldsymbol {\Sigma }})^{-{\frac {1}{2}}}\,e^{-{\frac {1}{2}}(\mathbf {x} -{\boldsymbol {\mu }})'{\boldsymbol {\Sigma }}^{-1}(\mathbf {x} -{\boldsymbol {\mu }})}\\
	\mathbb{E}X &= \mu \\
	Var[X] &= \Sigma
\end{align*}
\href{https://en.wikipedia.org/wiki/Multivariate_normal_distribution}{wikipedia}

\subsection{Empirical Distribution}
For any observation $X'=(x'_1, \cdots, x'_n)$, the empirical distribution is defined as:
\begin{align*}
	f(x) &= \hat{f}(x) =\frac{1}{n}\sum_{i=1}^{n}\delta(x - x_i)\text{, where $\delta$ is the dirac-delta function}\\
	\mathbb{E}X &= \hat{\mathbb{E}}X = \frac{1}{n}\sum_{i=1}^{n}x_i \\
	Var[X] &= \hat{Var}[X] =\frac{1}{n}\sum_{i=1}^{n}(x_i-\hat{\mathbb{E}}X)^2
\end{align*}
\href{http://www.stat.umn.edu/geyer/5102/slides/s1.pdf}{lecture}

\section{Estimation}
TODO: ML, Score-Function, biased/unbiased, Cramér–Rao bound, confidence-interval

\section{Divergences}
Conventions for this section: $P$ and $Q$ are probability measures over a set $X$, and $P$ is absolutely continuous with respect to $Q$. $S$ is a space of all probability distributions with common support.
\subsection{Divergence}
A divergence on $S$ is a function $D: S \times S \rightarrow R$ satisfying
\begin{enumerate}
	\item $D(p || q) \geq 0  \forall p, q \in S$,
	\item $D(p || q) = 0 \Leftrightarrow p = q$
\end{enumerate}
\textit{A divergence is a "sense" of distance between two probability distributions. It's not a metric, but a pre-metric.}\\
\href{https://en.wikipedia.org/wiki/Divergence_(statistics)}{wikipedia}

\subsection{f-Divergence}
\begin{enumerate}
	\item Generalization of whole family of divergences
	\item For a convex function $f$ such that $f(1) = 0$, the f-divergence of $P$ from $Q$ is defined as:\\
	$D_{f}(P\parallel Q)\equiv \int _{{\Omega }}f\left({\frac{dP}{dQ}}\right)\,dQ$
	\item \href{https://en.wikipedia.org/wiki/Divergence_(statistics)}{wikipedia}
\end{enumerate}

\subsection{KL-Divergence}
\begin{enumerate}
	\item The Kullback–Leibler divergence from $Q$ to $P$ is defined as\\
	$D_{\mathrm {KL} }(P\|Q)=\int _{X}\log {\frac {dP}{dQ}}\,dP=D_{t\log t}$.
	\item maxmizing likelihood is equivalent to minimizing $D_{KL}(P(. \vert \theta^{\ast}) \, \Vert \, P(. \vert \theta))$, where $P(. \vert \theta^{\ast})$ is the true distribution and $P(. \vert \theta)$ is our estimate.
	\item \href{https://en.wikipedia.org/wiki/Kullback–Leibler_divergence}{wikipedia}
	\item TODO: Fisher-Matrix infitesimal relationship
\end{enumerate}

\subsection{Jensen–Shannon divergence}
The Jensen–Shannon divergence from $Q$ to $P$ is defined as
\begin{align*}
	{{\rm {JSD}}}(P\parallel Q)={\frac  {1}{2}}D(P\parallel M)+{\frac  {1}{2}}D(Q\parallel M)
\end{align*},\\
 where $M={\frac  {1}{2}}(P+Q)$\\
\href{https://en.wikipedia.org/wiki/Jensen–Shannon_divergence}{wikipedia}

\subsection{TODO: Wasserstein \& Wasserstein Dual}

\section{Information Geometry}
TODO